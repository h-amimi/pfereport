
\chapter*{\centerline{Introduction générale}}
\phantomsection
\addcontentsline{toc}{chapter}{Introduction générale}

Le projet de fin d’étude marque la fin de ma formation d’ingénieur au sein de l’ENSA. Il représente ma possibilité à mettre en pratique les compétences et connaissances acquises durant mes cinq dernières années d’études. J’ai eu l’opportunité d’effectuer mon stage au sein de la société ALTEN dans le département Automobile à Fès.

Le secteur automobile est en pleine évolution et ne cesse de croître. La croissance du nombre de constructeurs et d’équipementiers en ai la preuve. Les systèmes mécaniques dans l’automobile ont été largement remplacé par des systèmes électroniques. Aujourd’hui, l’industrie automobile intègre des systèmes embarqués dans tout et n’importe quoi, partant des essuie-glaces jusqu’aux airbags et systèmes ABS.


C’est dans ce cadre qu’ALTEN intègre l’industrie de l’automobile, elle vise à renforcer son positionnement dans ce marché qui prend de l’ampleur.

Dans les années 2000, la crise du carburant a été l'une des causes principales de la croissance du développement des véhicules hybrides. Le Roadster de Tesla Motors, qui a été mis en vente en 2008, a révolutionné l'industrie. La conception attrayante et la gamme étendue du Roadster ont attiré un marché plus grand que jamais et ont encouragé des concurrents tels qu'\textbf{AUDI} à lancer leurs propres modèles.

Le but de ce projet sera la réalisation et l’automatisation des tests système du projet AWC (Automatic Wireless Charging) de \textbf{PRIMOVE BOMBARDIER}. Ce projet permettra aux véhicules électriques de charger leurs batteries avec un niveau d’efficacité très élevé et dans une période assez courte.

\noindent Ce rapport se répartira alors sur quatre chapitres :

\begin{itemize}
	\item Le premier chapitre ``\textit{Contexte général du projet}'' présentera dans un premier lieu l’organisme d’accueil ainsi que le projet sur lequel on va travailler.
	\item Le deuxième chapitre ``\textit{Environnement de projet}'' qui introduira l'environnement matériel ainsi que les outils logiciels utilisés tout au long du projet.
	\item Le troisième chapitre ``\textit{Description détaillée du projet}'' présentera en détails les différentes fonctions incorporées dans le système AWC, il expliquera aussi en profondeur la fonction \textbf{Sécurité} du système qui m'a été confié lors de ce stage.
	\item Enfin le dernier chapitre qui est le quatrième ``\textit{Test??}'' présentera le processus de création des spécifications de tests ainsi que le développement des scripts correspondants.
\end{itemize}