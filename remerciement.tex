
\chapter*{\centerline{Remerciement}}
\phantomsection
\addcontentsline{toc}{chapter}{Remerciement}

Il m’est agréable de m’acquitter d’une dette de reconnaissance auprès de toutes les personnes, dont l’intervention au cours de ce projet, a favorisé son aboutissement.

Mes très chers remerciements vont à \textbf{Monsieur Anass MANSOURI}, mon encadrant et professeur à l'Ecole Nationale des Sciences Appliqués de Fès pour m’avoir encadré et encouragé tout au long de ce projet, et prodigué ses directives précieuses et ses conseils pertinents qui m’ont été d’un appui considérable dans ma démarche. Qu'il trouve dans ce travail un hommage vivant à sa haute personnalité.

Je tiens à remercier vivement \textbf{Monsieur Hicham AMARA}, mon encadrant au sein d’ALTEN pour pour l'orientation, la confiance, la patience qui ont constitué un apport considérable sans lequel ce travail n'aurait pas pu être mené au bon port.

Mes remerciements les plus sincères vont aussi à \textbf{Monsieur Mohamed BOUHADDA}, résponsable du département Automobile et Aéronautique, de m’avoir accueilli et accepté parmi eux.

Merci aux membres du jury pour l'honneur qu'ils me font en jugeant ce travail.

Je tiens à remercier tout le personnel d'ALTEN, pour leurs soutiens et pour leurs générosités considérables quant à l’offre de l’information.

Je tiens également à adresser mes plus sincères remerciements à l’ensemble du corps enseignant de l’Ecole Nationale des Sciences Appliquées de Fès, pour avoir porté un vif intérêt à ma formation, et pour nous avoir accordés le plus clair de leur temps, leur attention et leur énergie et ce dans un cadre agréable de complicité et de respect.

Enfin, que tous ceux et celles qui ont contribués de près ou de loin à l’accomplissement de ce travail trouvent l’expression de mes remerciements et de ma considération.